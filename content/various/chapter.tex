\chapter{Various}

\section{Intervals}
	\kactlimport{IntervalContainer.h}
	\kactlimport{IntervalCover.h}
	\kactlimport{ConstantIntervals.h}

\section{Misc. algorithms}
	\kactlimport{TernarySearch.h}
	\kactlimport{LIS.h}
	\kactlimport{FastKnapsack.h}

\section{Dynamic programming}
	\kactlimport{KnuthDP.h}
	\kactlimport{DivideAndConquerDP.h}

\section{Debugging tricks}
	\begin{itemize}
		\item \verb@signal(SIGSEGV, [](int) { _Exit(0); });@ converts segfaults into Wrong Answers.
			Similarly one can catch SIGABRT (assertion failures) and SIGFPE (zero divisions).
			\verb@_GLIBCXX_DEBUG@ failures generate SIGABRT (or SIGSEGV on gcc 5.4.0 apparently).
		\item \verb@feenableexcept(29);@ kills the program on NaNs (\texttt 1), 0-divs (\texttt 4), infinities (\texttt 8) and denormals (\texttt{16}).
	\end{itemize}

\section{Misc}
	\verb@__builtin_ia32_ldmxcsr(40896);@ disables denormals (which make floats 20x slower near their minimum value).
	\subsection{Bit hacks}
		\begin{itemize}
			\item \verb@x & -x@ is the least bit in \texttt{x}.
			\item \verb@for (int x = m; x; ) { --x &= m; ... }@ loops over all subset masks of \texttt{m} (except \texttt{m} itself).
			\item \verb@c = x&-x, r = x+c; (((r^x) >> 2)/c) | r@ is the next number after \texttt{x} with the same number of bits set.
		\end{itemize}
	\subsection{Formulas}
		\begin{itemize}
		    \item Pick's Theorem: $A = i + \frac{b}{2} - 1$
		    \item Euler's formula: $vertices - edges + faces = 2$ (In connected planar graphs)
		    \item $Catalan_n = \frac{1}{n+1} \cdot \binom{2n}{n}$ = $\binom{2n}{n} - \binom{2n}{n+1} = \frac{2n!}{n!(n+1)!}$
		    \item $Catalan_n^{(k)} = \frac{k+1}{n+k+1} \cdot \binom{2n+k}{n}$ Count of balanced parentheses sequences consisting of n+k
                   pairs of parentheses where the first K
                   symbols are open brackets.
            \item Stirling Numbers of the First Kind:
                 \begin{itemize}
                    \item unsigned $s(n, k)$ counts the number of permutations of size n with $k$ cycles.
                    \item $s(n+1, k) = -n \cdot s(n, k) + s(n, k - 1)$ (for unsigned just $n$ instead of $-n$)
                    \item $s(0, 0) = 1$ and $s(0, n) = s(n, 0) = 0$ for $n > 0$.
                    \item $(x)_n = \sum\limits_{k=0}^n{s(n, k)\cdot{x^k}}$ where $(x)_n = x(x-1)(x-2)...(x-n+1)$, so the stirling numbers are the coefficients of the rising factorials polynomial.
                \end{itemize}
           \item Stirling Numbers of the Second Kind:
                 \begin{itemize}
                     \item Number of ways to partition a set of n objects into k non-empty subsets.
                     \item $S(n+1, k) = k\cdot{S(n, k)}+S(n, k-1)$
                     \item $S(n, k) = \frac{1}{k!} \sum\limits_{i=0}^k -1^i \binom{k}{i} (k-i)^n$
                 \end{itemize}
		\end{itemize}
	\kactlimport{FastMod.h}
	\kactlimport{FastInput.h}
	\kactlimport{BumpAllocator.h}
	\kactlimport{SmallPtr.h}
	\kactlimport{BumpAllocatorSTL.h}
	% \kactlimport{Unrolling.h}
	\kactlimport{SIMD.h}
